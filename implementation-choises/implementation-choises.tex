\documentclass[]{article}

%opening
\title{Scelte Implementative progetto RCL}
\author{Edoardo Ermini}

\begin{document}

\maketitle

\begin{abstract}

\end{abstract}

\section{Project}
\begin{itemize}
	\item Le liste \textbf{todo}, \textbf{inProgress}, \textbf{toBeRevised}, \textbf{done} sono implementate come tabelle hash.
\end{itemize}

\section{I/O}
Per la parte di lettura e scrittura di file e directories sul filesystem è stato scelto di utilizzare java.nio piuttosto che java.io, in particolare sono stati utilizzati i metodi statici delle classi Files e Paths.

\subsection{Paths}
La classe Paths contiene metodi statici per la creazione di istanze i Path.
L'interfaccia Path rimpiazza la classe File in java.io e rappresenta la locazione di un file o una directory.

\subsection{Files}
La classe Files contiente metodi statici per la manipolazione di files e directories.

\end{document}
