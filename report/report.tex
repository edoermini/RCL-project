%Carattere dimensione 11
\documentclass[11pt]{report}

%Margini e interlinea
\usepackage[margin=2cm]{geometry}
\pagestyle{plain}
\linespread{1}

%Librerie utili
\usepackage[italian]{babel}
\usepackage[utf8]{inputenc}
\usepackage{times}
\usepackage{graphicx}
\usepackage{floatflt}
\usepackage{blindtext}
\usepackage{enumitem}
\usepackage{amsthm}
\usepackage{subfig}
\usepackage{listings}
\usepackage{listingsutf8}
\usepackage{amsmath}
\usepackage{framed}
\usepackage{minibox}
\usepackage{float}
\usepackage{wrapfig}
\usepackage{longtable}
\usepackage[strict]{changepage}
\usepackage{pgfplots}
\usepackage{tikz}
\usepackage{titletoc}
\usepackage{hyperref}
\usetikzlibrary{matrix}
\pgfplotsset{width=11cm,compat=1.9}

\begin{document}
	
	\begin{titlepage}
		
		\linespread{2}
		
		\begin{figure}[t]
			\centering\includegraphics[width=0.9\textwidth]{unipi_logo}
		\end{figure}
		
		\begin{center}
			\vspace*{5mm}
			{\LARGE{\bf Reti di Calcolatori e Laboratorio Corso A}}\\
			\vspace{5mm}
			{\LARGE{\bf Relazione progetto WORTH}}\\
		\end{center}
		
		\vspace{10mm}
		
		\centering{\large{\bf Anno Accademico 2020/2021 }}
		
		\vspace{30mm}
		
		\hfill
		\begin{minipage}[t]{0.47\textwidth}\raggedright
			{\large{\bf Professoressa: \\ Laura Ricci\\ }}
		\end{minipage}
		\hfill
		\begin{minipage}[t]{0.47\textwidth}\raggedleft
			{\large{\bf Presentato da: \\ Edoardo Ermini\\ }}
		\end{minipage}
		
	\end{titlepage}

	\tableofcontents
	
	\chapter{Scelte progettuali}
	
	\section{Componenti}
	
	\subsection{Lato server}
		Le componenti base del sistema sono:
		
		\begin{itemize}
			\item \textbf{Card} rappresenta una card appunto
			
			\item \textbf{Project} rappresenta un progetto, contiene le card e i membri relativi al progetto
			
			\item \textbf{User} reppresenta un utente
		\end{itemize}
	
		Definite tutte nel package \textit{com.worth.components} e utilizzate dalle strutture condivise:
		
		\begin{itemize}
			\item \textbf{ProjectsManager} mantiene la lista di tutti i progetti, permette l'aggiunta, la modifica e la cancellazione di quest'ultimi.\\
			Utilizza le funzioni di rmi callback definite nell'iterfaccia \textbf{CallbackServer} per notificare agli utenti:
			\begin{itemize}
				\item[1] I nuovi progetti dei quali sono membri
				\item[2] I progetti eliminati e dei quali erano membri
			\end{itemize}
			Utilizza i metodi statici della classe \textbf{Writer} per rendere persistente lo stato del sistema.
			
			\item \textbf{UsersManager} mantiene la lista degli utenti, permette la registrazione il login e il logout di quest'ultimi.\\
			Utilizza, come sopra, le funzioni di rmi callback per notificare agli utenti il cambio dello stato di un utente (online/offline).\\
			Utilizza i metodi statici della classe \textbf{Writer} per rendere persistente lo stato del sistema.
			
		\end{itemize}
	
	\subsection{Lato client}
	Le componenti principali lato client sono 2:
	
	\begin{itemize}
		\item \textbf{Cli} definita in \textit{com.worth}, avvia la connessione TCP con il server e gestisce la comunicazione con il server.\\
		Mantiene le strutture dati modificate dalle funzioni di rmi callback chiamate dal server.
		
		\item \textbf{Chat} definita in \textit{com.worth.chat} è un thread avviato per ogni progetto dell'utente e contiene i metodi per leggere e scrivere da/sulla chat.
	\end{itemize}
		 
	
	\section{Persistenza}
	La persistenza dello stato del sistema è garantita e viene attuata attraverso le calssi \textbf{Reader} e \textbf{Writer} definite dento il package \textit{com.worth.io}, entrambe composte da soli metodi statici. \\
	Per permettere una consistenza e persistenza delle informazioni anche in casi eccezionali di guasti del server è stato deciso di scrivere sul filesystem dopo ogni modifica o aggiunta di un informazione in memoria.
	
	\subsection{Struttura database}
	Lo stato del sistema viene salvato all'interno della directory \textit{db} creata all'avvio del server nel caso fosse inesistente.\\
	Ogni progetto viene rappresentato con una directory scritta all'interno della directory \textit{db/projects}. \\
	Gli attributi \textit{chatIp} e \textit{members} di ogni progetto vengono salvati in 2 file distinti all'interno della directory \textit{nomeprogetto/.meta} nominati rispettivamente \textit{ip} e \textit{members}. \\
	Le cards vengono invece salvate all'interno della directory \textit{nomeprogetto} creando un file per ogni card. \\
	Gli utenti vengono salvati all'interno della directory \textit{db/users} creando come per le cards un file per ogni utente. \\
	Il salvataggio di dati non primitivi come array o in generale oggetti di del sistema viene fatto serializzando l'oggetto in un formato json e poi salvando la stringa json in un file, in particolare questo avviene per gli utenti, i membri di ogni progetto e le cards.
	
	\section{Sicurezza}
	È stato deciso di non mantenere le password in chiaro ne in memoria ne sul dico ma bensì al momento della registrazione viene calcolato e salvato l'hash della password inviata dal client.\\
	In questa maniera si garantisce lato server un minimo livello di sicurezza nel caso in cui ci fossero delle intrusioni e/o in casi di information disclosure. 
	
	\section{Chat}
	La chat è gestita interamente lato client con l'utilizzo dei metodi di callback implementati nella classe \textbf{ClientEvent} contenuta nel package \textit{com.worth.rmi.callback} e della classe \textbf{Chat} contenuta in \textit{com.worth.chat}. \\
	
	\subsection{Comunicazione ip delle chat}
	È stato scelto di usare i meccanismi di java RMI e java RMI Callback oltre che per la registrazione e per notificare ai vari clients connessi la lista degli utenti, anche per comunicare l'indirizzo ip di multicast per le chat di nuovi progetti o progetti a cui un utente è stato inserito come membro.\\
	In particolare l'interfaccia condivisa tra client e server \textbf{ClientEventInterface} definita nel package \textit{com.worth.rmi.callback} contiene i metodi:
	
		\begin{itemize}
			\item notifyProjectIp(String projectName, String ip) 
			\item notifyDeletedProject(String projectName)
		\end{itemize}

	Il primo serve per notificare agli utenti membri del progetto l'ip di quest'ultimo, il secondo per notificare invece sempre agli stessi la cancellazione di un progetto del quale erano membri, questo per permettere di eliminare lato client le risorse allocate per leggere la chat di quel progetto.\\
	L'unico momento in cui il client viene a conoscenza degli ip dei vari progetti senza il meccanismo di callback è al login dove, se l'operazione è avvenuta con successo, vengono inviati insieme alla lista degli utenti come risposta sulla connessione TCP avviata dal client.\\
	La scelta di utilizzare questo metodo è motivata dal fatto che la notifica di aggiunta e/o di cancellazione dei progetti è quasi istantanea e permette al client di essere sempre aggiornato su azioni di altri su progetti di cui è membro. \\
	In più evita all'utente di avviare esplicitamente la lettura di nuovi messaggi dalle chat dei progetti di cui è membro infatti tutto questo viene fatto automaticamente nell'implementazione lato client dell'interfaccia \textbf{ClientEventInterface}.
	
	\subsection{Lato client}
	\textbf{Chat} viene utilizzata per avviare un thread che in background legge e salva i messaggi che arrivano per un determinato progetto, contiene in oltre 2 metodi: uno per leggere i messaggi arrivati dall'ultima lettura e uno per inviare un messaggio in chat. \\
	Il metodo di callback \textit{notifyProjectIp} crea un instanza Chat per ogni nuovo progetto e avvia il thread che legge i messaggi per quel progetto, il metodo \textit{notifyDeletedProject} invece termina il thread avviato per un progetto eliminato e quindi stoppa il salvataggio di nuovi messaggi.
	
	\subsubsection{Eliminazione di un progetto}
	È stato scelto di permettere agli utenti di leggere la chat di un progetto anche dopo la sua eliminazione.\\
	In particolare quando un progetto viene eliminato l'operazione viene eseguita e notificata attraverso le callback ai vari utenti membri del progetto, e quindi lato client le risorse allocate per leggere la chat vengono eliminate. Quello che non viene eliminato è l'oggetto contenente i messaggi non letti, almeno fino al momento in cui l'utente non crea o non viene aggiunto a un progetto con lo stesso nome di quello eliminato.
	
	\subsection{Lato server}
	L'unica cosa che fa il server sulle chat dei vari progetti è inviare messaggi di notifica ai membri di un progetto al quale è stata fatta una modifica, come aggiunta di cards o membri, spostamento di cards e/o rimozione del progetto stesso.
	
	\section{Server timeout}
	Per evitare uno spreco di risorse è stato previsto un tempo massimo di attesa del server per un input da parte dell'utente impostato a 5 minuti. \\
	Se un utente non interagisce con il server per più di 5 minuti il server fa il logout dell'utente e chiude la communicazione con il client, a questo punto il client rimane in attesa di un input da parte dell'utente e non appena questo avviene si chiude stampando un messaggio di timeout. \\
	
	\section{Schema dell'architettura}
	\begin{center}
		\includegraphics[width=1\textwidth]{project-architecture}
	\end{center}
	
	\chapter{Scelte implementative}
	\section{Strutture dati}
	È stato scelto di rappresentare quasi tutte le liste con la struttura dati \textbf{HashMap} o la versione concorrente \textbf{ConcurrentHashMap} per rendere efficiente aggiunta, ricerca e rimozione di informazioni dato che vengono eseguite con complessità pari a $\Theta(1)$.\\
	
	\subsection{Lato server}
	Per le 4 liste delle card, per la lista dei progetti e degli utenti è stata usata un'\textbf{HashMap} con chiave rispettivamente il nome della card, del progetto o dell'utente e come valore l'oggetto corrispondente. Non c'è stato bisogno di una struttura concorrente perché ogni metodo di \textbf{ProjectsManager} e \textbf{UsersManager} che gestisce queste informazioni prende la lock implicita sull'oggetto prima di eseguire qualsiasi operazione.\\
	\textbf{ConcurrentHashMap} è stata usata invece per rappresentare gli utenti interessati a ricevere notifiche con callback in \textbf{CallbackServer} dato che i metodi \textit{registerForEvents} e \textit{unregisterForEvents} non fanno altro che eseguire rispettivamente una \textit{put} e una \textit{remove}.
	
	\subsection{Lato client}
	È stata usata una \textbf{ConcurrentHashMap} sia per la lista degli oggeti \textbf{Chat} che per la lista degli utenti in quanto vengono accedute contemporaneamente da \textbf{Cli} e dai metodi di callback.\\
	Per la prima viene usato come chiave il nome del progetto e come valore l'oggetto \textbf{Chat} corrispondente, per la seconda come chiave viene usato il nome dell'utente e come valore un booleano dove true equivale a utente online false il contrario.
	
	\section{Porte utilizzate}
	Per il corretto funzionamento di tutto il servizio sono state utilizzate 3 porte:
	\begin{itemize}
		\item 6660 la porta alla quale si connettono i clients
		\item 6661 la porta usata per il registro rmi
		\item 6662 la porta usata per le chat in multicast
	\end{itemize}
	
	\section{Modalità di gestione dei client}
	È stato scelto di implementare la gestione delle connessioni e delle richieste dei vari client con un multithreaded server piuttosto che un server con selector.
	Questo perché si è preferito mantiene una scalabilità elevata, ritardi di reply e accept generalmente bassi e una semplicità nell'implementazione andando a perdere nell'efficienza nell'I/O non utilizzando java NIO.
	
	\section{Gestione dei thread}
	Lato server vengono avviati tanti threads quanti sono i client che si collegano, lato client vengono avviati tanti threads quanti sono i progetti di cui quel determinato utente è membro.
	In entrambi i casi viene utilizzata una threadpool in particolare viene creata una chached threadpool dato che il numero di clients per il server e di progetti per il clients è variabile.
	
	\section{Thread safeness}
	È stato deciso di utilizzare il meccanismo dei monitor e alcune strutture concorrenti ove possibile per garantire semplicità nel codice e soprattutto perché non è stato possibile utilizzare lock esplicite e quindi sincronizzare i thread con una granularità più fine.
	
	\subsection{Lato server}
	Utilizzando un multithreaded server, per gerantire una consistenza delle informazioni create, modificate e rimosse dai client, è stato necessario rendere l'accesso agli utenti e ai progetti, quindi anche alle card e ai membri dei quest'ultimi mutualmente esclusivo.\\
	Le classi che gestiscono queste informazioni sono \textbf{UsersManager} e \textbf{ProjectsManager} e tutti i metodi di entrambe sono \textit{synchronized} dato che vanno a modificare lo stato che per la prima sono le liste degli utenti online e offline e per la seconda è la lista dei progetti. 
	
	\subsection{Lato client}
	Le uniche strutture condivise sono \textit{usersList} che mantiene le coppie $<nome utente, stato>$, \textit{chats} che mantiene le coppie $<nome progetto, chat>$ e \textit{threadPool} che mantiene tutti i thread che leggono le chat dei vari progetti dell'utente.\\ 
	Per le prime due è necessaria una sincronizzazione dato che vengono accedute contemporaneamente da \textbf{Cli} e dai metodi di \textbf{CallbackServer} e per l'ultima no dato che viene modificata solo dai metodi di callback, \textbf{Cli} la modifica solo dopo o prima che il client si è rispettivamente disiscritto o iscritto per le callback dal server.
	
	\section{Generazione degli ip di multicast per le chat}
	Per permettere la riusabilità degli ip dei progetti eliminati viene utilizzato l'\textbf{ArrayList} \textit{usedIPs} definito dentro \textbf{ProjectsManager} che salva gli ip in uso.
	La generazione di un ip per ogni nuovo progetto viene fatta dal metodo \textit{generateIP} sempre definito in \textbf{ProjectsManager} che genera un ip di multicast random controlla che non si trovi dentro \textit{usedIPs} e se questa condizione è verificata ritorna l'ip generato altrimenti reitera fino a quando non genera un ip non in uso.
	Quando un progetto viene cancellato si elimina anche da \textit{usedIPs} l'ip corrispondente a quel progetto permettendo il riuso.
	
	
	\section{Metodi della specifica}
	I metodi:
		\begin{itemize}
			\item \textit{createProject}
			\item \textit{cancelProject}
			\item \textit{addCard}
			\item \textit{showCard}
			\item \textit{showCards}
			\item \textit{moveCard}
			\item \textit{getCardHistory}
			\item \textit{addMember}
			\item \textit{showMembers}
			\item \textit{listProjects}
		\end{itemize}
	sono stati implementati lato server dentro la classe \textbf{ProjectsManager}.\\
	I metodi:
		\begin{itemize}
			\item \textit{login}
			\item \textit{logout}
			\item \textit{register}
		\end{itemize}
	sono stati implementati lato server dentro la classe \textbf{UsersManager}. \\
	Infine i metodi:
		\begin{itemize}
			\item \textit{sendChatMsg}
			\item \textit{readChat}
			\item \textit{listUsers}
			\item \textit{listOnlineUsers}
		\end{itemize}
	sono stati implementati lato client dentro la classe \textbf{Cli}.
	
	\subsection{Modifiche}
	È stata modificata la firma dei metodi:
		\begin{itemize}
			\item \textit{createProject}
			\item \textit{cancelProject}
			\item \textit{addCard}
			\item \textit{showCard}
			\item \textit{showCards}
			\item \textit{moveCard}
			\item \textit{getCardHistory}
			\item \textit{addMember}
			\item \textit{showMembers}
			\item \textit{sendChatMsg}
		\end{itemize}
	con l'aggiunta del parametro \textit{user} che corrisponde all'utente che richiede l'operazione, questo per controllare se effetivamente l'utente ha i permessi per fare quella determinata operazione che sta richiedendo e per sapere chi vuole inviare il messaggio nel caso dell'ultimo metodo.\\
	È stato modificato ulteriormente il metodo \textit{moveCard} eliminando il parametro che corrisponde alla lista di partenza della card, dato che non dava nessuna informazione in più non reperibile direttamente dal server. Infatti la lista in cui la card si trova prima del movimento è conosciuta dal server e ovviamente viene utilizzata, prima di eseguire lo spostamento, per controllare se quest'ultimo è fattibile.
	
	\chapter{Formato dei messaggi}
	Nei messaggi scambiati tra client e server viene usato il carattere \% per separare le varie informazioni, tutti i messaggi hanno il seguente formato:
	\begin{center}
		(request/response code)\%parametro1\%parametro2\%parametro3....
	\end{center}
	dove parametro1, parametro2, parametro3,... sono informazioni che dipendono dal tipo di richiesta o di risposta.
	
	\section{Messaggi di richiesta}
	\begin{center}
		\begin{tabular}{ |c|c|c| } 
			\hline
			Codice richiesta & operazione & parametri\\
			\hline\hline
		  	0	& login				& 
		  		\begin{tabular}{ c|c }
		  			username & password
		  		\end{tabular}\\ 
	  		\hline
			1	& logout			& username\\ 
			\hline
			2	& list projects		& \\
			\hline
			3	& create project	& progetto\\
			\hline
			4	& add member		& 
				\begin{tabular}{ l|r }
					progetto & membro
				\end{tabular}\\
			\hline
			5	& show members		& progetto\\
			\hline
			6	& show cards 		& progetto\\
			\hline
			7	& show card 		& 
				\begin{tabular}{ l|r }
					progetto & card
				\end{tabular}\\
			\hline
			8	& add card			& 
				\begin{tabular}{ l|c|r }
					progetto & card & descrizione
				\end{tabular}\\
			\hline
			9	& move card			& 
				\begin{tabular}{ l|c|r }
					progetto & card & destinazione
				\end{tabular}\\
			\hline
			10	& get card history	& 
				\begin{tabular}{ l|r }
					progetto & card
				\end{tabular}\\
			\hline
			11	& cancel project	& progetto\\
			\hline
		\end{tabular}
	\end{center}

	\section{Messaggi di risposta}
	Anche i messaggi di risposta iniziano con un codice che indica il successo o una classe di errore, nel caso del successo quello che segue il codice sono dei dati o una stringa che descrive l'avvenuto successo, nel caso di errore quello che segue il codice è un stringa che descrive l'errore.
	
	\begin{center}
		\begin{tabular}{ |c|c| } 
			\hline
			Codice risposta & significato \\
			\hline\hline
			0	& operazione eseguita con successo \\
			\hline
			1	& errori di sintassi \\
			\hline
			2	& errori relativi agli utenti/membri \\
			\hline
			3	& errori relativi alle passwords \\
			\hline
			4	& errori relativi a operazioni dove è necessario prima o dopo un login o logout \\
			\hline
			5	& errori relativi ai progetti \\
			\hline
			6	& errori di permessi per alcune operazioni \\
			\hline
			7	& errori relativi alle cards \\
			\hline
		\end{tabular}
	\end{center}

	\chapter{Compilazione e Avvio}
	
	\section{Dipendenze}
	L'unica dipendenza necessaria per compilare e runnare server e i clients è maven.\\
	L'installazione di maven varia in base al sistema operativo, per farlo correttamente si può seguire la guida del sito ufficiale andando su  \href{https://maven.apache.org/install.html}{https://maven.apache.org/install.html}.
	
	\subsection{Debian based}
	Se ci si trova su un sistema basato su debian, come ubuntu, l'intallazione è più semplice, infatti basta aprire il terminale e digitare:
	\begin{lstlisting}[language=bash]
	sudo apt install maven
	\end{lstlisting}

	
	\section{Run}
	Per compilare e avviare il server posizionarsi all'interno della directory \textit{worthserver} e digitare sul terminale:
	\begin{lstlisting}[language=bash]
	mvn compile
	mvn exec:java -Dexec.mainClass="com.worth.ServerMain"
	\end{lstlisting}
	
	Una volta avviato il server per avviare un client aprire un nuovo terminale, posizionarsi all'interno della directory \textit{worthclient} e digitare:
	\begin{lstlisting}[language=bash]
	mvn compile
	mvn exec:java -Dexec.mainClass="com.worth.WorthMain"
	\end{lstlisting}
	
	Si consiglia di non copiare e incollare dal documento ma di scrivere a mano i comandi perché in base al lettore di pdf che si usa i caratteri copiati potrebbero essere errati e dare quindi errori di compilazione.\\
	Una volta avviato il client per vedere la lista dei comandi disponbili digitare il comando \textit{help}.
	
	
	
	

	
	
	
	
	 
	
	
	 
	
	
	
	
	
	
\end{document}